\chapter*{Nomenklatur}
\addcontentsline{toc}{chapter}{Nomenklatur}
\paragraph{Abk\"urzungen}
\begin{table}[H]
	\centering
		\begin{tabular}{p{2cm}p{15cm}}
			ADC \dotfill & Analog/Digital Wandler\\
			ADP \dotfill & Advanced Design Project\\
			AEC \dotfill & Advanced Design Project\\
			ALU \dotfill & Arithmetische Logik Einheit\\
			ARM\dotfill & Advanced RISC Machines\\
			BF\dotfill & Bereichsforderung\\
			CAN\dotfill & Controller Area Network\\
			CPU\dotfill & central processing unit\\
			CSMA/CR\dotfill & carrier sense multiple access/collision resolution\\
			EMI\dotfill & Elektromagnetische Interferenz\\
			EMV\dotfill & Elektromagnetische Verträglichkeit\\
			FCLK\dotfill & Prozessorkerntakt\\
			FF\dotfill & Festforderung\\
			FPU\dotfill & Gleitkommaeinheit\\
			FPGA\dotfill & Field Programmable Gate Array\\
			GND\dotfill & Ground\\
			HSE\dotfill & High Speed External\\
			IC\dotfill & Integrated Circuit\\
			ID\dotfill & Identifier\\
			IEEE\dotfill & Institute of Electrical and Electronics Engineers\\
			IMS\dotfill & Institut für Mechatronische Systeme\\
			IPC\dotfill & Association Connecting Electronics Industries\\
			JTAG\dotfill & Joint Test Action Group\\
			LDO\dotfill & Low Dropout\\
			MCU\dotfill & microcontroller unit\\
			MOSFET\dotfill & Metall-Oxid-Halbleiter-Feldeffekttransistor\\
			NKW\dotfill & Nutzkraftwagen\\
			NTC\dotfill & Negative Temperature Coefficient\\
			PC\dotfill & Personal Computer\\
			PCB\dotfill & printed circuit board\\
			PID\dotfill & proportional-integral-derivative (controller)\\
			PKW\dotfill & Personenkraftwagen\\
			PLCD\dotfill & permanentmagentic linear contactless displacement\\
			PWM\dotfill &  Pulsweitenmodulation\\
			RAM\dotfill & random access memory\\
			ROM\dotfill & read only memory\\
			RTC\dotfill & real-time clock\\
			SMD\dotfill & surface-mounted device\\
			SRAM\dotfill & Static random-access memory\\
			STM\dotfill & STMicroelectronics\\
			SWD\dotfill & Serial Wire Debugging\\
			THT\dotfill & through hole technology\\
			UART\dotfill & Universal Asynchronous Receiver Transmitter\\
			VCC\dotfill & Versorgerspannung\\
			W\dotfill & Wunsch \\
			ZF\dotfill & Zielforderung\\
		\end{tabular}
		\label{Abkuerzung}
\end{table}


\paragraph{Griechische Symbole}
\begin{table}[H]
\centering
				\begin{tabular}{p{1.5cm}p{13.5cm}p{2cm}}
			$\beta$ \dotfill& Beta-Wert des Thermistors &  $ K $\\
			$\kappa$ & Proportionalitätsfaktor des IS-Pins & -\\
			$\rho$ \dotfill& spezifischer Widerstand & \SI{}{\Omega m}\\
			$\varrho$ \dotfill& Dichte & \SI{}{\frac{kg}{m^3}}\\
			$\tau$ \dotfill& Periodendauer &  s\\
		\end{tabular}
		\label{Abkurzung}
\end{table}

%\newpage
\paragraph{Lateinische Symbole}
\begin{table}[H]
	\centering
		\begin{tabular}{p{1.5cm}p{13.5cm}p{1.5cm}}
			A \dotfill& Leiterquerschnitt & \SI{}{m^2}\\
			B \dotfill& magnetische Flussdichte & T\\
			b \dotfill& Leiterbahnbreite & m\\
			C \dotfill& Kapazität & F\\
			$\text{C}_w$ \dotfill& Wärmekapazität & \SI{}{\frac{J}{K}}\\
			c \dotfill& spezifische Wärmekapazität & \SI{}{\frac{J}{KgK}}\\
			$\text{C}_{ADC}$ \dotfill& Kapazität des ADC & F\\
			d \dotfill& Dicke der Kupferschicht & m\\
			$\text{F}_L$ \dotfill& Lorentzkraft & N\\
			$\text{f}_{ADC}$ \dotfill& Frequenz des ADC & Hz\\
			I \dotfill& Strom & A\\
			$\text{I}_{L}$ \dotfill& Laststrom & A\\
			$\text{I}_{IS}$ \dotfill& Strom am IS-Pin & A\\
			k \dotfill& Anzahl Messtakte pro Messung & -\\
			l \dotfill& Leiterlänge & m\\
			m \dotfill& Masse & kg\\
			N \dotfill& Auflösung & -\\
			P \dotfill& Leistung & W\\
			p \dotfill& Tastverhältnis & -\\
			$\text{Q}_w$ \dotfill& Wärmeenergie & J\\
			R \dotfill& Widerstand & $\Omega$ \\
			$\text{R}_{AIN}$ \dotfill& Eingangswiderstand & $\Omega$\\
			$\text{R}_{ADC}$ \dotfill& Schaltwidertand des ADC & $\Omega$\\
			$\text{R}_{R}$ \dotfill& Widerstand bei Referenztemperatur & $\Omega$ \\
			T \dotfill& Temperatur& $^\circ$C/K\\
			$\text{T}_R$ \dotfill& Referenztemperatur & $^\circ$C/K\\
			$\text{t}_{aus}$ \dotfill& Ausschaltzeit & s\\
			$\text{t}_{ein}$ \dotfill& Einschaltzeit & s\\
			$\text{U}_a$ \dotfill& aktuelle Spannung & V\\
			$\text{U}_b$ \dotfill& Betriebsspannung & V\\
			$\text{U}_m$ \dotfill& zeitlicher Mittelwert der Spannung & V\\
			$\text{U}_0$ \dotfill& Spannung bei $\text{x}=0$& V\\
			$\text{U}_1$ \dotfill& Spannung am linken Anschlag & V\\
			$\text{U}_2$ \dotfill& Spannung am rechten Anschlag & V\\
			x \dotfill& Schaltgabelposition & m\\
			$\text{x}_{ges}$ \dotfill& Wegdifferenz zwischen Anschlagpositionen & m\\
			
			
			\end{tabular}	
\end{table}

