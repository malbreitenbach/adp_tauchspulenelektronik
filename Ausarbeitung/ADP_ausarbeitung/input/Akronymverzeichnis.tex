\chapter*{Nomenklatur}\addcontentsline{toc}{chapter}{Nomenklatur}
\section*{Abkürzungen}
ADC - Analog/Digital Wandler\\
ADP – Advanced Design Project\\
AEC - Automotive Electronics Council\\
ALU - Arithmetische Logik Einheit\\
ARM - Advanced RISC Machines
BF – Bereichsforderung\\
CAL - Kalibrierung des Lagesensors\\
CAN – Controller Area Network\\
CAN_RECEIVE - empfangsseitige Controller Area Network-Schnittstelle\\
CAN_TRANSMIT - sendeseitige Controller Area Network-Schnittstelle\\
CONH - Regler zum Positionshalten\\
CONS - Regler zum Schalten\\
CPU - central processing unit\\
CSMA/CR - carrier sense multiple access/collision resolution\\
ECU - Steuereinheit (\textit{electronic control unit}\\
EMI - Elektromagnetische Interferenz\\
EMV - Elektromagnetische Verträglichkeit\\
ERR - Fehlererkennung\\
FCLK - Prozessorkerntakt
FF – Festforderung\\
FPU - Gleitkommaeinheit\\
FPGA - Field Programmable Gate Array\\
GND - Ground\\
HSE - High Speed External\\
IC - Integrated Circuit\\
ID - Identifier\\
IEEE - Institute of Electrical and Electronics Engineers\\
IMS – Institut für Mechatronische Systeme\\
IPC - Association Connecting Electronics Industries\\
JTAG - Joint Test Action Group\\
LDO - Low Dropout\\
MAIN - Hauptzustandsautomat\\
MCU – microcontroller unit\\
MOSFET - Metall-Oxid-Halbleiter-Feldeffekttransistor\\
MOT - Motortreiber\\
NKW - Nutzkraftwagen\\
NTC - Negative Temperature Coefficient\\
PC - Personal Computer\\
PCB - printed circuit board\\
PID - proportional-integral-derivative (controller)\\
PKW - Personenkraftwagen\\
PLCD – permanentmagentic linear contactless displacement\\
POS - Schaltgabelpositionsermittlung\\
PWM - Pulsweitenmodulation\\ 
PWMS - Auswahlglied für pulsweitenmoduliertes Signal\\
RAM - random access memory\\
ROM - read only memory\\
RTC - real-time clock\\
SEN - Sensorik\\
SMD - surface-mounted device\\
SRAM - Static random-access memory\\
STM - STMicroelectronics\\
SWD - Serial Wire Debugging\\
THT - through hole technology\\
UART - Universal Asynchronous Receiver Transmitter
VCC - Versorgerspannung
W - Wunsch\\
ZF – Zielforderung\\
